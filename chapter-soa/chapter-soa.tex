
\chapter{Requirements Space}

\chapterintro*

\blindtext

The use of industrial wireless networks has been studied in many works in the literature. However, no comprehensive survey of the whole problem space of industrial communications has been performed.

In \cite{6248648}, the authors have introduced a comparison between the commercial and industrial communications networks where an industrial network has been divided to five different levels. These levels include field equipment, controller level, application, supervisory, and external networks. The differences in requirements between different levels are discussed. Moreover, three types of information are considered which are control, diagnostic, and safety information as described in \cite{4118467}. However all these levels of industrial networks are mentioned in \cite{6248648}, the article focuses only on the manufacturing and instrumentation communications and does not consider other types of communications networks that exist in industrial environments. Also, in \cite{What}, three levels of communications are considered which are device, control, and information levels. Moreover, the current wired industrial technologies for these levels are discussed briefly.

More works focused on the communications at the field devices level where sensing and control information is transfered. In \cite{7005074},  the communication between field devices has been studied where the requirements for a large number of nodes may not be achieved. The use of fieldbus solutions limit the scalability and resilience and hence industrial Ethernet capabilities are introduced in this article. Moreover, in \cite{Connectivity}, the communication for monitoring and control operations is discussed. A comparison between fieldbus technologies, industrial Ethernet, and wireless solutions is performed. The author has discussed the use of Wi-Fi, Bluetooth, ZigBee, and WirelessHART technologies in industrial applications. Similarly, the authors of \cite{6490786} considered the industrial communications networks requirements in process automation specifically at field devices level. Finally, in \cite{GE_Professional}, many case-studies are discussed for communication networks in industrial scenarios. Moreover, the design steps for these solutions are briefly discussed.

\chapter{Systems Modeling}

\chapterintro*

\blindtext

Current modeling work on factory work-cells is mainly aimed at defining and characterizing the subsystems, such as human staff, robots, and machine tools, in individual applications. By following blueprints (schematics) of production tasks, the work flow can be divided into separate assignments which are distributed by a task dispatch system to individual machines~\cite{IkeaBot}. Analytical models are thus obtained for performance analysis in work-cells. As an example, a mathematical model for real-time performance analysis of a gantry work-cell with robots is established with the timing and the randomness of tasks and disruptions are captured  \cite{8098604}. In \cite{OU2017212}, the same model is used to investigate the system natural properties such the system cycle and waiting times and to identify bottlenecks through studying the sensitivity of each machine. Similarly, the steady state analysis for production lines with uncertainties is performed through various decomposition methods~\cite{Colledani2013,doi:10.1080/00207543.2012.713137,doi:10.1080/00207540500385980}. In \cite{Colledani2013}, a decomposition method is presented for the analysis of continuous flow lines. The presented model is used to analyze flow lines with single and multiple failure mode machines and machines subject to aging and having up and down times. In \cite{doi:10.1080/00207543.2012.713137}, a model to evaluate the performance of transfer lines with unreliable machines and finite transfer-delay buffers is presented. A decomposition method is introduced to model the transfer line, using the general-exponential distributions instead of the exponential distributions to approximate the repair time distributions of the fictitious machines. In \cite{doi:10.1080/00207540500385980}, the authors present a model for evaluating the production rate and distribution of inventory of a closed-loop manufacturing system with unreliable machines and finite buffers. The model accounts for the different sets of machines that could cause blockage or starvation to other machines. In \cite{QChang,Liu2012}, the performance analysis modeling for serial production lines with disruptions is explored by studying the impact of each individual downtime event in terms of permanent production loss and financial cost. These analytical models generally work well for simple systems with small number of components or few interactions between various equipment. Also, the analytical models can be used to abstract industrial systems to understand various performance trends without studying various details. As a result, we introduce a comprehensive model that include network and production impacts on the industrial work-cell.  

Furthermore, the reconfigurable work-cell architecture is widely considered for automated manufacturing. The main advantage of reconfigurable work-cells lies in the flexibility of reconfiguration of work-cell components to adapt to varying production requirements where the assembly of the work cell is optimized for each specific task~\cite{CHEN2001199}. In the work-cell that hosts robots, robots are installed therein to allow for autonomous configuration within their workspace \cite{8023523,10.1007/978-3-319-65151-4_10,6059204}. Approaches and performance criteria for reconfigurable robotic systems have recent developments in control architectures to achieve various levels of reconfigurability \cite{Fulea}. The National Institute of Standards and Technology (NIST) has defined a Network of Things (NoT) model which can depict the structure of work-cells by a group of NoT building blocks and model the behaviors of individual components in a work-cell~\cite{NIST800-183}.  The NIST NoT model is focused primarily on sensor networks and the collection of data.  Actuation is cursorily noted, and, as such, cross-domain interactions between the physical system and the network are not addressed. Several other robotic work-cell architectures are discussed in the literature. In \cite{OpenArch}, a reference model for a control system functional architecture applied to open architecture robot controllers is presented. In \cite{CARPANZANO2007435}, a methodology to develop self-adaptive factory automation solutions is illustrated, using a novel modular simulation based method. With the increase in complexity and reconfigurability of work-cells, studying various production criteria and networks impacts requires introducing new models to capture these interactions and to be abstract enough to model different configurations and scenarios of industrial work-cells.  

In a work-cell model, data flows are used to capture the trajectory of system information exchange between work-cell components and identify their roles in specific operations~\cite{OpenArch}. For example, safety-related operations employ the vision system and various proximity sensors that generate proximity data and transmit them to the safety manager to define safety zones in an automotive assembly work-cell~\cite{safeeye}. 
In another example, data flows are enabled in a work-cell to capture human operator gestures from embedded cameras in human-robot collaborations~\cite{cobotcell}. These gestures can be later regenerated in simulators based on the transmitted position data from the field to optimize work-cell safety operations~\cite{gesture}. Currently, most of the work-cell information in these scenarios are transmitted by wired networks. Wireless networks have gained increasing interests to enable data flows in the highly connected work-cells. Wireless standard bodies have proposed their network reference models in factory environments which include the work-cell cases in the data-centric architecture~\cite{ETSI889, KPItable}. In these models, individual work-cells are treated as a subnetwork of field instruments attached with data aggregations that manage network connections and transfer data traffic to edge and cloud servers in various applications. Wireless connections are featured with flexible network topology to agree with a variety of transmission needs, especially in reconfigurable work-cells. Meanwhile, data traffic flows are characterized by select performance metrics, such as transmission latency and link reliability, to categorize industrial use cases~\cite{KPItable}. 

Current modeling efforts set the boundaries of their systems of study at the edge devices without further discussions on the impact of wireless performance on the operations of industrial systems. For example, the abstracted disruptions in \cite{QChang,Liu2012} that cause plant downtimes may include wireless network impacts which are not yet treated distinguishably with specific characteristics of wireless networks. As indicated by the earlier empirical studies~\cite{LIU2017412}, such physical systems may have different responses to network performance which will vary with the operational configuration such as the served ``application'' and the deployed control algorithms. In this paper, we incorporate the features of wireless communications networks into the modeling architecture of physical work-cells such that cross-domain interactions may be studied.  Prior to introducing the model for wireless incorporation, we first provide the reader an introduction to SysML in the following section.

%\section{Systems Modeling Using SysML} \label{sysml:sec:systemsmodeling}

The goal of modeling a system is the capture of knowledge of a process in a simplified way~\cite{SysModel2004}. A secondary goal of a system model is to provide a level of abstraction that may allow for the discovery of new knowledge such as how two systems will interact. There are multiple ways of designing and presenting system models. Well-behaved systems can be represented by a system of equations using mathematical tools~\cite{SimModel1999}. Such models provide excellent constraint definition, but lack the semantics to describe architecture and detailed information flow.
Moreover, by deploying functional block diagrams, we are able to capture major functional components and flow of information or material. As shown in Fig.~\ref{sysml:fig:fbd-system}, a physical process interacts with a control system through a wireless network.  Measured values, $Y$, from sensors flow to the controller through a wireless network and arrive at the controller delayed and modified, $\bar{Y}$. Similarly, commands, $U$, flows from the controller to the actuators through the wireless network. Such diagrams may be used to model feedback control systems in which the origination and routing of information are immaterial for study. However, the architecture and interfaces remain at a very high level of abstraction making analysis difficult. In such cases, delay and loss using such tools are often modeled stochastically. Using architectural diagrams helps identify components, interfaces, and information flow. For factory systems, architectural block diagrams are often manifested as schematics. However, such diagrams have their own limits in industrial practices. On one hand, they lack the semantics necessary to describe the constraints that formal equations and functional block diagrams offer. Meanwhile, they also lack the capability of capturing behaviors or complex interactions between the physical system and the information infrastructure such as a wireless network. 

\begin{figure}
	\centering
	\includegraphics[width=0.65\columnwidth]{./chapter-sysml/diagrams/fbd-system}
	\caption{Functional block diagram of a cyber-physical system in which a physical process and an automation system interact through a wireless network.}
	\label{sysml:fig:fbd-system}
\end{figure}

An alternative to schematic diagrams is SysML~\cite{SysML2017}. SysML is a general purpose modeling language that is often used for model-based systems engineering (MBSE) practice within industrial systems~\cite{MBSEandSysML}. SysML provides structural, behavioral, and parametric semantics for the analysis of complex systems. For examples, systems analysis using SysML enables capturing and communicating system requirements and design which include hardware, software, firmware, information flows, and processes with graphical notations. Within the factory automation industry, engineers are adopting SysML in the form of MBSE to develop realizable operational models of the factory and data flow processes. MBSE models address verification of design through executable simulations depending on the modeling tool.  The SysML specification is defined in~\cite{SysML2017}.  In SysML, the basic semantic constructs of the language are Packages, Blocks, Ports, Interfaces, and Constraints, in addition to the constructs provided by the Unified Modeling Language (UML). Packages are logical grouping of model elements. Package relationships are captured using the package diagram (PKG). Relationships of these constructs are captured in the block definition diagram (BDD).  The internal composition and connectivity of parts are captured in the internal block diagram (IBD). SysML includes other types of diagrams and semantic constructs that are not required for this analysis and are not explained here. The SysML model is comprehensive; however, the size and number of diagrams within the model are too extensive to include within this paper.  Therefore, the reader is encouraged to explore the SysML model defined in~\cite{SysML.Candell2018}.  A useful primer on SysML may be found in~\cite{Friedenthal2015.SysML}. 

Examples of the use cases and methodologies of using different graphical models for the analysis of manufacturing systems are explored in~\cite{Lutjen2015.GramosaMethod,Luder2011.GraphicalModeling,Jia2013.GraphicalModeling,Alvarez2013.GraphicalModeling}.  In~\cite{Quinsat2017.SysML}, SysML is used to capture both composition and behavior of an additive manufacturing work-cell.  A survey of applying graphical modeling languages in capturing information flows within a product service system which may be applied to manufacturing enterprises~\cite{Durugbo2011.GraphicalModeling}.  Our approach compliments these previous examples by combining the operational and wireless information transport systems together in a single model, thereby facilitating a single model that may be used for simulation and other systems engineering analyses.

While various architectures for the work-cell exist as exemplified in the literature, a common language and framework for communicating architecture and information flow has not been established for cross-domain interactions between the manufacturing system and its supporting communication networks.
SysML contains the semantics for such engineering capture and provides an industry accepted language for communicating composition, interfaces, and information flow. Moreover, SysML provides the semantics for assigning properties to any model element such that those properties are made  available for analysis using other tools such as Prot\'eg\'e \cite{StanfordUniversity.Protege} and the Web Ontology Language (OWL) \cite{W3C2012.OWL}.
It is important to understand that while SysML provides semantics for a formal capture of architecture, information flow, and parametric constraints, it may also be used for a higher-degree of abstraction provided by the functional block diagrams.  

\chapter{Use of Databases for Performance Analysis}

\chapterintro*

\blindtext

Multiple surveys about GDBs have been presented to describe the associated models, tools, and their features in~\cite{Angles:2008:SGD:1322432.1322433,7148480,GDB_overview}. Also, examples of applications and implementations of GDBs are presented in~\cite{modern_models} to show their use on enterprise data, social networks, and determining security and access rights. It was found that GDBs provide the much needed structure for storing data and incorporating a dynamic schema. On the other hand, query languages are used to extract data including traversing the database, comparing nodes properties, and subgraph matching~\cite{Wood2012QueryLF}. The performance of different GDB tools and methodologies is analyzed and compared in~\cite{Jadhav2015ComparativeAO,Macko:2013:PIG:2485732.2485750}. Multiple comparisons in these articles have shown improved performance of Neo4j in the general features for data storing and querying, and data modeling features such as data structures, query languages and integrity constraints. 

Furthermore, industrial data analytics play an essential role in achieving the smart factory vision and improving decision-making in various industrial applications. Five main industrial data methodologies are studied including highly distributed data ingestion, data repository, large-scale data management, data analytics, and data governance~\cite{DBLP:journals/corr/abs-1807-01016}. Industrial data processing offers valuable information about various sections of industrial applications including inefficiencies in industrial processes, costly failures and down-times, and effective maintenance decisions~\cite{JLee}. In~\cite{4}, a platform for performing industrial big data analysis is presented where the performance requirements are introduced to achieve a cost-effective operation. Various other frameworks for industrial data analysis can be found in~\cite{5,6}, where the importance of using data analysis in decision making is emphasized.

Due to its advantages including scalability, efficiency, and flexibility, NoSQL databases are a popular alternative to relational databases in the case of large amounts of data in various applications~\cite{doi:10.1108/17440081311316398}. The GDB is a kind of NoSQL database approaches that additionally handles complex relationships~\cite{8123475}. GDBs are widely adopted in various industry-related applications and use cases such as network operations, fraud detection, and asset and data management~\cite{top5}. Relationships in social networks have been modeled using a GDB for structural information mining and marketing~\cite{Gomez-Rodriguez:2012:IND:2086737.2086741}. On the other hand, business solutions for scenarios with multiple large data sources require distributed processing in decision making for various problems such as fraud detection, trend prediction, and product recommendation~\cite{Skhiri2013}.


\chapter{Machine Learning in Performance Estimation}

\chapterintro*

\blindtext

In the literature, two types of interference signals are considered, namely, intentional or unintentional interference. Methods to estimate, avoid, or mitigate interference are required for the deployment of reliable and deterministic IWSs. Machine learning has been widely used to detect and estimate interference information to enhance the performance of interference managment algorithms.  

The interference analysis in cyber-physical systems (CPSs) has been considered in multiple works for various scenarios. In~\cite{8639006}, in-network interference mitigation techniques are discussed for ultra reliable low-latency wireless communications systems. The paper focused on mutual interference mitigation in an industrial automation setting, where multiple transmissions from controllers to actuators interfere with each other. In \cite{Kumar2019}, an interference mitigating receiver architecture is proposed. The application scenarios are smart homes and modern factories where dense wireless communications devices exist. Moreover, in~\cite{Bhushan2014.NetworkDesensUsingIntfCancellation}, interference cancellation of transmissions from neighboring cells in a 5G cellular network is presented. In~\cite{Gomes2017.LQEinWSNs}, a method using a dedicated node for link quality estimation (LQE)  obtained through received data packets to identify interference and multi-path without introducing additional traffic is presented. In~\cite{Baccour2012.SurveyOnLinkEstimation}, a taxonomy of channel link quality techniques is presented providing a valuable survey on LQE algorithms and asserting the importance of link quality estimation in IWSs. In~\cite{Frounhoffer.Troubleshooting}, failure analysis and wireless network troubleshooting are performed whenever the CPS is not functioning properly. Interference analysis is one major part of the troubleshooting procedure which is performed through traffic patterns and wireless spectrum analysis. Also, in~\cite{NIST.InterfCoexCritical}, the use of spectrum analysis for interference detection and estimation is proposed for IWSs. 

On the other hand, intended interference (i.e., jamming) can lead to service denial or poor performance in wireless networks. In~\cite{8631535}, a literature review was presented which includes an overview of recent research efforts on networked control systems under denial-of-service attacks such as jamming attacks in wireless channels. One of the discussed challenges is how to achieve ultra-reliable low-latency "signalling" within industrial applications. In~\cite{Cetinkaya_2019}, a discussion is also provided on the recent developments concerning the design of attack-resilient control and communication protocols. Generally, a jamming attacker can block transmission of packets by emitting strong interference signals to a wireless channel \cite{1637931,5473884}. Jamming attacks can target various wireless technologies and hence can become a major concern for control systems, since they are easy to launch \cite{5473884}. It was shown in \cite{10.1007/978-3-319-07788-8_40} that off-the-shelf hardware can be used for generating jamming attacks on wireless networks. In cases of physical-layer attacks, the jamming attacker targets a frequency band and is not required to follow the wireless protocol where it can cause a decrease in the SIR thus preventing the receiver from successfully detecting transmitted packets \cite{10.1007/978-3-319-07788-8_40}. In the case of MAC-layer attacks, both the packet sender and the jamming attacker operate on the same channel; the jamming attacker’s goal is to cause packet collisions.
In~\cite{8726803}, the authors evaluated the CPSs resilience to jamming attacks that disrupt wireless communications. They considered three jamming strategies which are the constant, random, and protocol-aware jamming. They showed through experimental results that various CPS control schemes are susceptible to constant and random jamming while the time-triggered control schemes are susceptible to protocol-aware jamming. Moreover, resilience of CPSs is also considered in~\cite{6425868,DEPERSIS2014134,7402971,7575630} where periodic jamming is considered in \cite{6425868} while the jamming strategy in~\cite{DEPERSIS2014134,7402971,7575630} is neither known nor pre-fixed. 

Machine learning has been used for detection and estimation of jamming attacks. In~\cite{Chen2019}, an unsupervised machine learning algorithm based on a multi-layer autoencoder is used to extract the interference source spectrum features. These features are then used to distinguish interference sources type and location without labeling measured data. In~\cite{7911887},  an unsupervised approach using a recurrent neural network to detect anomalies in the CPS performance and identify attacked sensors. In~\cite{Junejo2016DataDP}, a behavior based machine learning intrusion detection approach  is proposed to detect attacks at the physical process layer. The results are validated through experimental study of a real modern water treatment facility. In~\cite{Beaver:2013:EML:2584691.2584722}, the viability of machine learning methods in detecting the new threat scenarios of command and data injection is assessed. In that work, command and control communications in a critical infrastructure setting are monitored and vetted against examples of benign and malicious command traffic to identify potential attack events. In~\cite{6900095}, the authors assessed discriminating types of power system disturbances through machine learning by detecting jamming attacks. They evaluated various machine learning methods as disturbance discriminators and discuss the practical implications for deploying machine learning systems as an enhancement to existing power system architectures.

Therefore, in the literature, it was shown that LQE is one important but insufficient aspect of assessing the impact of link quality on a CPS. We assert that by jointly observing the performance of the physical and wireless components of a CPS,the complete perspective of the quality of the wireless link and its impact on physical performance can be obtained. Since interference is such an important topic in the wireless CPS, we are motivated to propose a method that simultaneously (1) makes observations of the physical system using ground truth measurements, and (2) infers the quality of the wireless communication system in terms of SIR using an experimental model of a relevant use case found in industry.
